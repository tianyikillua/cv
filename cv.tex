\documentclass[10pt,a4paper,ragged2e,withhyper]{altacv}

\usepackage{paracol}

\usepackage[rm]{roboto}
\usepackage[defaultsans]{lato}
\renewcommand{\familydefault}{\sfdefault}

\geometry{left=1.25cm,right=1.25cm,top=1.5cm,bottom=1.5cm,columnsep=1.2cm}

\begin{document}
\name{Tianyi \underline{LI}}
\tagline{Research engineer with 10+ years of experience in advanced simulation methods, bridging data-driven modeling and physics-based mechanics across scales}
\photo{2.5cm}{photo.jpg}
\personalinfo{
  \email{tianyikillua@gmail.com}
  \location{Montrouge, France}
  \printinfo{\faGoogleScholar}{\href{https://scholar.google.com/citations?user=jnOm13oAAAAJ}{Tianyi Li}}
  \linkedin{tianyikillua}
  \github{tianyikillua}
}

\makecvheader

%% Depending on your tastes, you may want to make fonts of itemize environments slightly smaller
\AtBeginEnvironment{itemize}{\small}

%% Set the left/right column width ratio to 6:4.
\columnratio{0.6}

% Start a 2-column paracol. Both the left and right columns will automatically
% break across pages if things get too long.
\begin{paracol}{2}
  \cvsection{Experiences}

  \cvevent{Simulation Technology Specialist}{\href{https://www.3ds.com}{Dassault Systèmes}, Corporate Research}{Jan 2020 -- now}{Vélizy-Villacoublay, France}
  \begin{itemize}
    \item \textbf{Deep Material Network} for multiscale material modeling: novel network architectures for parameterized microstructures, multiphysics property prediction, incrementally affine formulation for nonlinear behaviors, \textbf{PyTorch} and \textbf{UMAT} (C++) implementations, industrialization with R\&D, integration into both \textbf{CATIA} and \textbf{SIMULIA} products
    \item \textbf{Data-driven (model-free) computational mechanics}: theoretical and numerical investigations, extensions to nonlinear and inelastic behaviors, manifold learning, fixed-point acceleration
    \item \textbf{Physics-Informed Neural Network}: early research for solid mechanics applications using \textbf{PyTorch}, mixed formulations
    \item \textbf{GPU-based voxel solver} for simulation-driven design (C++, Python): fictitious domain method, matrix-free approach, geometric multigrid for linear elasticity and (transient) heat transfer, iterative linear solvers, industrialization with R\&D
    \item \textbf{Partitioned multiphysics coupling} methods (C++, Python): results mapping, temporal interpolation, dynamic mode decomposition surrogates, fluid-structure interaction simulations using \textbf{OpenFOAM} and \textbf{CalculiX}
    \item Development of various \textbf{ParaView Python plugins} for real-time simulation data interaction and visualization
    \item Patent drafting; journal and conference paper publications; participation in academic and industrial conferences
  \end{itemize}

  \divider

  \cvevent{Research and Development Engineer}{\href{https://web.archive.org/web/20210323140621/https://promold.fr/recherche-et-developpement}{Promold}}{Apr 2017 -- Dec 2019}{Paris, France}
  \begin{itemize}
    \item Injection molding (process) and integrative multiscale (structural) simulations of fiber-reinforced polymers with \textbf{Moldflow}, \textbf{Moldex3D}, \textbf{Optistruct}, \textbf{Radioss}, \textbf{code\_aster} and \textbf{Digimat}
    \item \textbf{Rheological modeling} of fiber-reinforced composites: anisotropic fiber-dependent viscosity and fiber orientation evolution. Implementation via \textbf{Moldflow Solver API} in C++
    \item Numerical tools development (Python) for results \textbf{mapping}, \textbf{mean-field homogenization} of fiber composites and \textbf{uncertainty propagation} using data-driven surrogates
    \item Adaptive optimization methodology of fiber orientation model parameters using \textbf{Kriging} and \textbf{Expected Improvement}
    \item Buckling analysis of fiber-reinforced materials with the finite element library \textbf{FEniCS} and eigenvalue solver \textbf{SLEPc}
    % \item Process automation under \textbf{HyperWorks} using \textbf{TCL}; \textbf{Docker} deployment; post-processing of simulation results under \textbf{ParaView}; data analysis and visualization using \textbf{Jupyter}
  \end{itemize}

  \divider

  \cvevent{PhD Candidate in Solid Mechanics}{\href{https://imsia.ip-paris.fr}{IMSIA (CNRS-EDF-ENSTA research lab)}}{Oct 2013 -- Sep 2016}{Palaiseau, France}
  \begin{itemize}
    \item \textbf{Phase-field fracture} modeling of brittle materials: variational formulation and numerical simulations (\href{https://pastel.archives-ouvertes.fr/tel-01487449}{\faLink\textbf{PhD thesis}})
    \item Implementation in \textbf{FEniCS} (Python) and in an industrial explicit dynamics code \textbf{Europlexus} (Fortran) using \textbf{PETSc}: quasi-perfect scaling obtained
    \item Contributions (C++) to the open-source finite element library \textbf{FEniCS}
  \end{itemize}

  %% Switch to the right column. This will now automatically move to the second
  %% page if the content is too long.
  \switchcolumn

  \cvsection{Most proud of}

  \cvachievement{\faCodeBranch}{\href{https://support.3ds.com/knowledge-base/?q=docid:QA00000420397}{Recent integration of the Deep Material Network model into Abaqus}}{thanks to continuous efforts and collaboration with SIMULIA teams}

  \divider

  \cvachievement{\faBullhorn}{My speech in front of 900 people}{and engagement with \href{https://eqdifference.org/qui-sommes-nous/nosrepresentants}{Eloquence de la Différence} as a volunteer and treasurer}

  \cvsection{Strengths}

  \cvtag{Nonlinear mechanics} \cvtag{Computational mechanics} \cvtag{Scientific machine learning} \cvtag{Programming} \cvtag{CAE tools} \cvtag{Scientific communication} \\ \cvtag{Listening and empathy}

  \cvsection{Typical day at work}

  \wheelchart{1cm}{0.2cm}{
    1/4em/accent!80/{Play with equations},
    1/5em/accent!70/{Read papers},
    1.5/6em/accent!60/{Coding},
    1/7em/accent!50/{Brainstorming},
    1/8em/accent!40/{Draw beautiful graphs},
    1/8em/accent!30/{Write reports},
    0.5/8em/accent!20/{Guide new minds}
  }

  \cvsection{Languages}

  \cvskill{Chinese}{5}

  \divider

  \cvskill{French / English}{4}

  \cvsection{Education}

  \cvevent{PhD in Solid Mechanics}{\href{https://www.polytechnique.edu}{Univ. Paris-Saclay (Ecole Polytechnique)}}{2013 -- 2016}{Palaiseau, France}

  \divider

  \cvevent{Engineer in Mechanics}{\href{https://www.utc.fr}{Université de Technologie de Compiègne}}{2010 -- 2013}{Compiègne, France}

  \divider

  \cvevent{Bachelor in Mechanics}{\href{https://utseus.shu.edu.cn/en.htm}{Université de Technologie Sino-Européenne de Shanghai}}{2007 -- 2010}{Shanghai, China}

\end{paracol}

\end{document}
